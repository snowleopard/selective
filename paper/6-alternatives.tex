\section{Alternative formulations}\label{sec-alternatives}

One might argue that the \hs{select} method should simply be added to the
\hs{Applicative} type class, with the default implementation
\hs{select}~\hs{=}~\hs{selectA}. This is indeed a viable approach, however, it
would make it harder to reason about code with the \hs{Applicative}~\hs{f}
constraint, since the \hs{select} method makes it possible for "new applicative"
effects to depend on values; declaring such a significant ability by the
\hs{Selective}~\hs{f} constraint would arguably be a more prudent approach.

Use \hs{branch} instead of \hs{select}, or use both
...
multi-way version, i.e. select1, select2, selectN, etc. up to infinite version