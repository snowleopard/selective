\section{Alternative formulations and further research opportunitites}
\label{sec-alternatives}

\subsection{Adding \hs{select} to \hs{Applicative}}\label{sec-alt-applicative}

One might argue that the \hs{select} method should simply be added to the
\hs{Applicative} type class, with the default implementation
\hs{select}~\hs{=}~\hs{selectA}. This is indeed a viable approach, however, it
would make it harder to reason about code with the \hs{Applicative}~\hs{f}
constraint, since the \hs{select} method makes it possible for "new applicative"
effects to depend on values; declaring such a significant ability by the
\hs{Selective}~\hs{f} constraint would arguably be a more prudent approach.

\subsection{Using \hs{branch} instead of \hs{select}}\label{sec-alt-branch}

Use \hs{branch} instead of \hs{select}, or use both
...
multi-way version, i.e. select1, select2, selectN, etc. up to infinite version

\subsection{Making \hs{Either} explicit}\label{sec-alt-either}

Arseniy's raise/catch

\subsection{Multiway \hs{select} and \hs{SelectiveDo}}\label{sec-selectivedo}
