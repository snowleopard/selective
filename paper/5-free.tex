\section{Free selective functors}\label{sec-free}

Free construction with examples

\subsection{Free construction}\label{sec-free-construction}

...

\subsection{Build systems, freely}

...

\subsection{Analysis and simulation of processor instructions}

The methodology of building effectful computations with free constructions such
as free~\cite{free-monads} and freer~\cite{freer-monads} monads and free
applicatives~\cite{free-applicatives} is a widespread in the functional programming community.
It allows to focus on the internal aspects of the effect under consideration and receive the
desired applicative/monadic structure of the computation~\emph{for free}, i.e. without the need
to construct~\hs{Applicative}/~\hs{Monad} instances and proving the laws.

In the ``free-structure'' methodology, the essence of an effect is a datatype which encodes
the ``commands''which the effect provides. This datatype acts as a deep embedding of the effect's
interface. THis datatype must only have enough structure to be a~\hs{Functor}. The purpose of
the free constructions is then to build on this functor a richer structure, which would have
an instance of Applicative/Selective/Monad.

In this section, we demonstrate how we can use free selective functors to construct an
effect which can be used for effectively describing the semantics of a simple instruction set
architecture. The features of free selective functors will allow for multiple distinct
interpretations of the same semantics, such as~\emph{static} dependency analysis
and~\emph{dynamic} simulation.

\subsubsection{Embedding}

We will represent the semantics of instruction in terms of the following datatype:

\begin{minted}{haskell}
type ISA a = Select RW a
\end{minted}

Here, \hs{Select} is the free selective functor defined earlier in this section.
We apply the \hs{Select} type constructor to the \hs{RW} datatype, which is the
functor we build our free construction on:

\begin{minted}{haskell}
data RW k v a = R k                 (v -> a)
              | W k (Program k v v) (v -> a)
    deriving Functor
\end{minted}

The effect we require comprises two commands. We need to have an ability to (1)
\emph{read} a value associated with a key and, (2) given a computation which produces a value,
\emph{write} its result into the store. Here, the second argument of the \hs{W} constructor
This exact structure of the definition is required for accommodating a pattern that
frequently occurs in instruction semantics: often we read a value from a location
(register/memory), do something with the value and then write it into a different location.
If we had the type of \hs{W} to be \hs{k -> v -> (v -> a)}, i.e. required the value to be pure,
we would not be able to get away from using monadic bind/join.





